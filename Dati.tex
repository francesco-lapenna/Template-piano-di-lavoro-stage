%----------------------------------------------------------------------------------------
%   USEFUL COMMANDS
%----------------------------------------------------------------------------------------

\newcommand{\dipartimento}{Dipartimento di Matematica ``Tullio Levi-Civita''}

%----------------------------------------------------------------------------------------
% 	USER DATA
%----------------------------------------------------------------------------------------

% Data di approvazione del piano da parte del tutor interno; nel formato GG Mese AAAA
% compilare inserendo al posto di GG 2 cifre per il giorno, e al posto di 
% AAAA 4 cifre per l'anno
\newcommand{\dataApprovazione}{22 Aprile 2025}

% Dati dello Studente
\newcommand{\nomeStudente}{Francesco}
\newcommand{\cognomeStudente}{Lapenna}
\newcommand{\matricolaStudente}{2072134}
\newcommand{\emailStudente}{francesco.lapenna.1@studenti.unipd.it}
\newcommand{\telStudente}{+ 39 392 60 45 612}

% Dati del Tutor Aziendale
\newcommand{\nomeTutorAziendale}{Alessandro}
\newcommand{\cognomeTutorAziendale}{Brighente}
\newcommand{\emailTutorAziendale}{alessandro.brighente@unipd.it}
\newcommand{\telTutorAziendale}{+ 39 000 00 00 000}
\newcommand{\ruoloTutorAziendale}{}

% Dati dell'Azienda
\newcommand{\ragioneSocAzienda}{Azienda S.p.A}
\newcommand{\indirizzoAzienda}{Via Roma 1, Roma (RM)}
\newcommand{\sitoAzienda}{http://example.com/}
\newcommand{\emailAzienda}{mail@mail.it}
\newcommand{\partitaIVAAzienda}{P.IVA 12345678999}

% Dati del Tutor Interno (Docente)
\newcommand{\titoloTutorInterno}{Prof.}
\newcommand{\nomeTutorInterno}{Stefano}
\newcommand{\cognomeTutorInterno}{Cecconello}

\newcommand{\prospettoSettimanale}{
     % Personalizzare indicando in lista, i vari task settimana per settimana
     % sostituire a XX il totale ore della settimana
     \begin{itemize}
        \item \textbf{Prima Settimana (40 ore)}
        \begin{itemize}
            \item Incontro per discutere i requisiti e le richieste relativi al sistema da sviluppare;
            \item Formazione sulle tecnologie adottate, inclusa la piattaforma OpenVLC e le schede BeagleBone Black.
            \item Studio dello standard IEEE 802.15.7 per le comunicazioni ottiche wireless a corto raggio;
            \item Studio di esempi già esistenti di autenticazione tramite luce visibile;
        \end{itemize}
        \item \textbf{Seconda Settimana (40 ore)} 
        \begin{itemize}
            \item Configurazione iniziale delle schede BeagleBone Black per la comunicazione tramite luce visibile.
            \item Analisi dei requisiti funzionali e non funzionali del modulo di autenticazione;
        \end{itemize}
        \item \textbf{Terza Settimana (40 ore)} 
        \begin{itemize}
            \item Progettazione dell'architettura del modulo di autenticazione, definendo i componenti principali e le loro interazioni;
            \item Documentazione delle scelte progettuali effettuate;
            \item Inizio dello sviluppo del modulo di autenticazione sulla piattaforma OpenVLC.
        \end{itemize}
        \item \textbf{Quarta Settimana (40 ore)} 
        \begin{itemize}
            \item Continuazione dello sviluppo del modulo di autenticazione;
            \item Implementazione delle funzionalità principali del modulo;
            \item Test preliminari per verificare il corretto funzionamento delle funzionalità implementate.
        \end{itemize}
        \item \textbf{Quinta Settimana (40 ore)} 
        \begin{itemize}
            \item Continuazione dello sviluppo del modulo di autenticazione;
            \item Implementazione di eventuali funzionalità aggiuntive richieste;
            \item Test approfonditi per verificare la sicurezza e la robustezza del sistema.
        \end{itemize}
        \item \textbf{Sesta Settimana (40 ore)} 
        \begin{itemize}
            \item Validazione del modulo sviluppato attraverso scenari realistici;
            \item Analisi dei risultati dei test e risoluzione di eventuali criticità;
            \item Revisione della documentazione tecnica relativa al modulo.
        \end{itemize}
        \item \textbf{Settima Settimana (40 ore)} 
        \begin{itemize}
            \item Preparazione della documentazione finale, comprensiva di risultati ottenuti e criticità riscontrate;
            \item Revisione del lavoro svolto con il tutor aziendale e il tutor interno;
            \item Eventuali miglioramenti o modifiche richieste.
        \end{itemize}
        \item \textbf{Ottava Settimana - Conclusione (40 ore)} 
        \begin{itemize}
            \item Consegna della documentazione finale;
            \item Presentazione del lavoro svolto ai responsabili del progetto;
        \end{itemize}
    \end{itemize}
}

% Indicare il totale complessivo (deve essere compreso tra le 300 e le 320 ore)
\newcommand{\totaleOre}{320}

\newcommand{\obiettiviObbligatori}{
    \item \underline{\textit{O01}}: Configurare le schede BeagleBone Black per consentire la comunicazione tramite luce visibile.
    \item \underline{\textit{O02}}: Studiare lo standard IEEE 802.15.7 per comprendere i requisiti tecnici e le specifiche.
    \item \underline{\textit{O03}}: Progettare e implementare il modulo di autenticazione sulla piattaforma OpenVLC.
    \item \underline{\textit{O04}}: Testare e validare il modulo sviluppato attraverso scenari realistici.
    \item \underline{\textit{O05}}: Documentare dettagliatamente il lavoro svolto e i risultati ottenuti.
}

\newcommand{\obiettiviDesiderabili}{
    \item \underline{\textit{D01}}: ?
    \item \underline{\textit{D02}}: ?
    \item \underline{\textit{D03}}: ?
}

\newcommand{\obiettiviFacoltativi}{
    \item \underline{\textit{F01}}: ?
    \item \underline{\textit{F02}}: ?
    \item \underline{\textit{F03}}: ?
}